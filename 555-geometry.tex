\documentclass{article}
\usepackage[brazil]{babel}
\usepackage[utf8]{inputenc}
\usepackage[T1]{fontenc}
\usepackage{lmodern}
\usepackage{soulutf8} 
\setlength{\headheight}{14.5pt}
\usepackage[a4paper,left=2.0cm,right=1.0cm,top=\dimexpr15mm+1.5\baselineskip,bottom=2cm]{geometry}
\usepackage{amsmath}
\usepackage{tikz}
\usepackage{mathdots}
\usepackage{yhmath}
\usepackage{cancel}
\usepackage{color}
\usepackage{wasysym}
\usepackage{siunitx}
\usepackage{array}
\usepackage{xargs}
\usepackage{multirow}
\usepackage{amssymb}
\usepackage{tabularx}
\usepackage{extarrows}
\usepackage{xfrac}
\usepackage{booktabs}
\usetikzlibrary{fadings}
\usetikzlibrary{patterns}
\usetikzlibrary{shadows.blur}
\usetikzlibrary{shapes}
\usepackage[utf8]{inputenc}
\usepackage{amsfonts}
\usepackage{amsmath}
\usepackage{multicol}
\usepackage{fancyhdr}
\usepackage{tikz, tkz-euclide}
\usetikzlibrary{patterns}
\usepackage{enumitem}
\usepackage{arcs}
\usepackage{afterpage}
\usepackage{tcolorbox}
\usepackage{multirow}
\usepackage{numprint}
\usepackage{titlesec}
\usepackage{graphicx}
\usepackage{pgfplots}
\pgfplotsset{compat=1.15}
\usetikzlibrary{patterns}
\pgfplotsset{compat=1.15}
\usepackage{mathrsfs}
\usetikzlibrary{calc,arrows,matrix}
\usetikzlibrary{shapes.geometric}
\usetikzlibrary{positioning,quotes}
\definecolor{BLA}{rgb}{0.26666666666666666,0.26666666666666666,0.26666666666666666}
\definecolor{WHI}{rgb}{255,255,255}

\titleformat{\section}[block]{\Large\bfseries\filcenter}{\thesection}{1em}{\ul}
\titleformat{\subsection}{\large\bfseries\filcenter}{\thesubsection}{1em}{}

\newcommand{\limite}[3][x]{\lim\limits_{{#1}\to{#2}}{#3}}

\newcommand{\sgn}{\hspace{2pt}\mathrm{sgn}\hspace{1pt}}

\newcommand{\pint}[1]{\lfloor{#1}\rfloor}

\newcommand{\deriv}[1]{{#1}^{\prime}}
\newcommand{\segderiv}[1]{{#1}^{\prime\prime}}

\newcommand{\integ}[3]{\int\limits_{#1}^{#2}{#3}}

\newcommand{\fimd}[1]{\hspace{2pt}d{#1}}

\newcommand{\multi}[7]{\textbf{({#1})} {#2}
\begin{enumerate}
    \item {#3}
    
    \item {#4}
    
    \item {#5}
    
    \item {#6}
    
    \item {#7}
\end{enumerate}}

\newcommand{\parenth}[1]{\left({#1}\right)}

\newcommand{\opcaoIMG}[2][0.5]{}

\pgfdeclarepatternformonly{south west lines}{\pgfqpoint{-0pt}{-0pt}}{\pgfqpoint{6pt}{6pt}}{\pgfqpoint{6pt}{6pt}}{
        \pgfsetlinewidth{0.16pt}
        \pgfpathmoveto{\pgfqpoint{0pt}{0pt}}
        \pgfpathlineto{\pgfqpoint{6pt}{6pt}}
        \pgfpathmoveto{\pgfqpoint{5.6pt}{-0.4pt}}
        \pgfpathlineto{\pgfqpoint{6.4pt}{0.4pt}}
        \pgfpathmoveto{\pgfqpoint{-0.4pt}{5.6pt}}
        \pgfpathlineto{\pgfqpoint{0.4pt}{6.4pt}}
        \pgfusepath{stroke}}
        
\makeatletter
\DeclareFontFamily{U}{tipa}{}
\DeclareFontShape{U}{tipa}{m}{n}{<->tipa10}{}
\newcommand{\arc@char}{{\usefont{U}{tipa}{m}{n}\symbol{62}}}%

\newcommand{\arc}[1]{\mathpalette\arc@arc{#1}}

\newcommand{\arc@arc}[2]{
	\sbox0{$\m@th#1#2$}%
	\vbox{
		\hbox{\resizebox{\wd0}{\height}{\arc@char}}
		\nointerlineskip
		\box0
	}%
}
\makeatother

\DeclareFontFamily{U}{skulls}{}
\DeclareFontShape{U}{skulls}{m}{n}{ <-> skull }{}
\newcommand{\skull}{\text{\usefont{U}{skulls}{m}{n}\symbol{'101}}}

\pagestyle{fancy}
\fancyhf{}
\fancyhfoffset[L]{0.5cm} % left extra length
\fancyhfoffset[R]{0.5cm} % right extra length
\lhead{MATH CHRONIC}
\chead{\bfseries 555 problemas de Geometria}
\rhead{Jeferson Almir}
\rfoot{}
\cfoot{\thepage}

\providecommand{\sin}{}
\renewcommand{\sin}{\hspace{2pt}\mathrm{sen}\hspace{1pt}}
\providecommand{\tan}{}
\renewcommand{\tan}{\hspace{2pt}\mathrm{tg}\hspace{1pt}}
\providecommand{\cot}{}
\renewcommand{\cot}{\hspace{2pt}\mathrm{cotg}\hspace{1pt}}
\providecommand{\csc}{}
\renewcommand{\csc}{\hspace{2pt}\mathrm{cossec}\hspace{1pt}}
\providecommand{\arctan}{}
\renewcommand{\arctan}{\hspace{2pt}\mathrm{arctg}\hspace{1pt}}
\providecommand{\arcsin}{}
\renewcommand{\arcsin}{\hspace{2pt}\mathrm{arcsen}\hspace{1pt}}
\newcommand{\arccot}{\hspace{2pt}\mathrm{arccotg}\hspace{1pt}}

\newcommand{\espaco}{$ $}
\newcommand{\dica}{\textbf{Dicas:}}

\newcommand{\iniTri}{Seja $ABC$ um triângulo}

\definecolor{ffqqqq}{rgb}{1.,0.,0.}
\definecolor{qqqqff}{rgb}{0.,0.,1.}
  
  
  
\begin{document}
\begin{center}
Professor: Jeferson Almir
\end{center}

Aluno(a): \underline{\hspace{12cm}}
Nº: \underline{\hspace{2cm}}

Data: \underline{\hspace{2cm}}/\underline{\hspace{2cm}}/\underline{\hspace{2cm}}

\begin{multicols}{2}

\section{Problemas}

\renewcommand{\labelenumi}{\textbf{\arabic{enumi}.}}
\renewcommand\labelenumi{\textbf{\nplpadding{3}\numprint{\arabic{enumi}}.}}

\begin{enumerate}

    \item Seja $ABC$ um triângulo. Prove que suas medianas $CD$, $AE$ e $BF$ são concorrentes. \dica %Áreas %Proporção %Ponto fantasma
    
    \item Seja $ABC$ um triângulo. Prove que suas alturas $AE$, $CF$ e $BD$ são concorrentes. \dica %Quadriláteros cíclicos %Ponto fantasma
    
    \item Prove que as bissetrizes internas de um $\triangle ABC$ são concorrentes. \dica %Definição
    
    \item Seja $ABC$ um triângulo. Seu incírculo toca $AB$, $BC$ e $CA$ nos pontos $C_1$, $A_1$ e $B_1$ respectivamente. Prove que as retas $CC_1$, $BB_1$ e $AA_1$ são concorrentes. \dica %Teorema de Ceva
    
    \item Prove que as mediatrizes dos lados de um dado $\triangle ABC$ são concorrentes. \dica %Definição
    
    \item Seja $ABC$ um triângulo de circuncírculo $k$. Sejam $l_A$, $l_B$ e $l_C$ as retas tangentes a $k$ pelos pontos $A$, $B$ e $C$ respectivamente. Se $l_A\cap l_B=C_1$, $l_B\cap l_C=A_1$ e $l_C\cap l_A=B_1$, prove que as retas $AA_1$, $BB_1$ e $CC_1$ são concorrentes. \dica %Teorema de Ceva %Analise outro triângulo
    
    \item Seja $ABC$ um triângulo. Sejam $A_1$, $B_1$ e $C_1$ os pontos de tangência dos segmentos $BC$, $CA$ e $AB$ com os exincírculos de $\triangle ABC$. Prove que as retas $AA_1$, $BB_1$ e $CC_1$ são concorrentes. \dica %Teorema do Bico %Teorema de Ceva
    
    \item Seja $ABC$ um triângulo e seja $N$ seu ponto de Nagel (ponto de concorrência do exercício anterior). Digamos que $AN$, $BN$ e $CN$ intersectem o incírculo de $\triangle ABC$ nos pontos $A_1$, $B_1$ e $C_1$, e os lados $BC$, $CA$ e $AB$ nos pontos $A_2$, $B_2$ e $C_2$, respectivamente. Prove que $AA_1=NA_2$, $BB_1=NB_2$ e $CC_1=NC_2$. \dica %Ponto fantasma %Tente achar uma homotetia útil %Teorema de Menelaus
    
    \item Seja $ABC$ um triângulo. Os triângulos equiláteros $\triangle ABC_1$, $\triangle AB_1C$ e $\triangle A_1BC$ são construídos no exterior do triângulo $ABC$. Prove que as retas $AA_1$, $BB_1$ e $CC_1$ são concorrentes. \dica %Marque ângulos %Procure quadriláteros cíclicos %Prove colinearidade
    
    \item \iniTri. Os triângulos equiláteros $\triangle ABC_1$, $\triangle AB_1C$ e $\triangle A_1BC$ são construídos no interior do triângulo $ABC$. Prove que as retas $AA_1$, $BB_1$ e $CC_1$ são concorrentes. \dica %Marque ângulos %Procure quadriláteros cíclicos %Prove colinearidade
    
    \item Prove que para um dado $\triangle ABC$, existe algum ponto $X$ tal que vale $AX\cdot BC = BX\cdot AC = CX \cdot AB$. \dica %Círculo de Apolônio %Eixo radical %Teorema de Menelaus
    
    \item Prove que para um dado $\triangle ABC$, exatamente dois pontos satisfazem a condição da questão anterior. \dica %Círculo de Apolônio %Eixo radical %Teorema de Menelaus
    
    \item \iniTri. Prove que existe um ponto único $S$ tal que vale $BC+AS = CA+BS = AB+CS$. \dica %Construa o incírculo do triângulo %Marque medidas %Construa circunferências secantes ao incírculo %Brinque com circunferências tangentes
    
    \item \iniTri. Prove que existe um ponto único $S$ tal que vale $BC-AS = CA-BS = AB-CS$. \dica %Construa o incírculo do triângulo %Marque medidas %Construa circunferências secantes ao incírculo %Brinque com circunferências tangentes %Pense fora da caixa
    
    \item Três circunferências $k_1(A)$, $k_2(B)$ e $k_3(C)$ são dadas, e elas são todas tangentes externamente entre si. Seja $C_1$ e $B_1$ os pontos de tangência de $k_1$ com $k_2$, e de $k_1$ com $k_3$, respectivamente. Seja $A_1$ o ponto de tangência de $k_2$ com $k_3$. A circunferência $k_4$ toca as outras três circunferências externamente. Prove que o primeiro centro de Soddy do $\triangle ABC$ (problema $13$) coincide com o centro de $k_4$. \dica %Faça uma correspondência de construções
    
    \item Três circunferências $k_1(A)$, $k_2(B)$ e $k_3(C)$ são dadas, e elas são todas tangentes externamente entre si. Seja $C_1$ e $B_1$ os pontos de tangência de $k_1$ com $k_2$, e de $k_1$ com $k_3$, respectivamente. Seja $A_1$ o ponto de tangência de $k_2$ com $k_3$. A circunferência $k_4$ toca as outras três circunferências internamente. Prove que o segundo centro de Soddy do $\triangle ABC$ (problema $14$) coincide com o centro de $k_4$. \dica %Faça uma correspondência de construções
    
    \item \iniTri. Sejam $S_1$ e $S_2$ seus primeiro e segundo centros de Soddy (problemas $13$ e $14$), respectivamente. Prove que os pontos $A$, $B$ e $C$ estão sobre uma elipse de focos $S_1$ e $S_2$. \dica %Use a definição desses pontos %Equacione
    
    \item \iniTri. Seja $S_1$ seu primeiro centro de Soddy (problema $13$). Prove que existe uma circunferência inscrita no quadrilátero convexo formado pelas retas $CS_1$, $BS_1$, $AC$ e $AB$. \dica %Construa o incírculo do triângulo %Marque medidas %Construa circunferências secantes ao incírculo %Equacione
    
    \item \iniTri. Seja $S_2$ seu segundo centro de Soddy (problema $14$). Prove que existe uma circunferência que toca as retas $BA$ e $BC$ e os segmentos $AS_2$ e $CS_2$. \dica %Construa o incírculo do triângulo %Marque medidas %Construa circunferências secantes ao incírculo %Tente achar uma versão do Teorema de Pitot para quadriláteros não-convexos %Equacione
    
    \item \iniTri, com exincírculos $\omega_a$, $\omega_b$ e $\omega_c$. Sejam $I_a$, $I_b$ e $I_c$ os centros de $\omega_a$, $\omega_b$ e $\omega_c$ respectivamente. Seja $A_1$ o ponto de tangência de $\omega_a$ com o lado $BC$. Defina os pontos $B_1$ e $C_1$ analogamente. Prove que as retas $C_1I_c$, $B_1I_b$ e $A_1I_a$ são concorrentes. \dica %Construa o incírculo do triângulo %Faça projeções do incentro sobre os lados %Estude as simetrias da construção
    
    \item \iniTri. O primeiro ponto de Brocard $Br_1$ é definido como o ponto para o qual $\angle BABr_1=\angle ACBr_1= \angle CBBr_1$. Prove que esse ponto sempre existe. \dica %Construa uma circunferência que passa por dois vértices e é tangente a um dos lados do triângulo %Marque ângulos
    
    \item \iniTri. O segundo ponto de Brocard $Br_2$ é definido como o ponto tal que $\angle ABBr_2 = \angle CABr_2 = \angle BCBr_2$. Prove que ele sempre existe. \dica %Construa uma circunferência que passa por dois vértices e é tangente a um dos lados do triângulo %Marque ângulos
    
    \item \iniTri. Seja $L$ seu ponto de Lemoine (problema $6$) e sejam $Br_1$ e $Br_2$ seu primeiro e segundo pontos de Brocard, respectivamente (problemas $21$ e $22$). Seja $CL\cap AB=F$. Prove que $\angle AFBr_1=\angle BFBr_2$. \dica %Teorema de Ceva Trigonométrico %Muita trigonometria
    
    \item \iniTri, com circuncentro $O$. Seja $L$ seu ponto de Lemoine (problema $6$) e sejam $Br_1$ e $Br_2$ seu primeiro e segundo pontos de Brocard, respectivamente (problemas $21$ e $22$). Prove que valem as igualdades $\angle OBr_1L=\angle OBr_2L=90^{\circ}$ e $Br_1L=Br_2L$. \dica %Projete o circuncentro sobre $AL$, $BL$ e $CL$ %Procure quadriláteros cíclicos
    
    \item \iniTri. Sejam $Ap_1$ e $Ap_2$ seus dois pontos isodinâmicos (problemas $11$ e $12$). Prove que os triângulos pedais com respeito a esses dois pontos são equiláteros. \dica %Projete os pontos sobre os lados do triângulo %Aplique Lei dos Senos
    
    \item \iniTri. Sejam $E$ e $D$ os pés das bissetrizes interna e externa em relação a $C$, respectivamente. Prove que os dois pontos isodinâmicos de $\triangle ABC$ (problemas $11$ e $12$) ficam sobre a circunferência de diâmetro $ED$. \dica %Use os teoremas de bissetrizes %Aplique Lei dos Senos %Teorema de Menelaus %Procure círculos de Apolônio %Centro radical
    
    \item \iniTri. Seja $T_1$ seu primeiro ponto de Fermat-Torricelli (problema $9$). Prove que $\angle AT_1B=\angle BT_1C=120^{\circ}$. \dica %Marque ângulos %Procure quadriláteros cíclicos %Prove colinearidade
    
    \item \iniTri. Seja $T_2$ seu segundo ponto de Fermat-Torricelli (problema $10$). Prove que vale exatamente uma das igualdades $\angle AT_2B=\angle AT_2C=60^{\circ}$, $\angle BT_2A=\angle BT_2C=60^{\circ}$ e $\angle CT_2B=\angle CT_2A=60^{\circ}$. \dica %Verifique a posição do ponto em relação ao triângulo %Marque ângulos %Procure quadriláteros cíclicos %Prove colinearidade
    
    \item \iniTri. Prove que o primeiro ponto isodinâmico (problema $11$) é conjugado isogonal do primeiro ponto de Fermat-Torricelli (problema $9$) com respeito a $\triangle ABC$. \dica %Use coordenadas baricêntricas %Determine as coordenadas baricêntricas do conjugado isogonal de um ponto de coordenadas baricêntricas $(u,v,w)$ %Faça uma razão entre as coordenadas baricêntricas dos pontos %Trigonometria
    
    \item \iniTri. Prove que o segundo ponto isodinâmico (problema $12$) é conjugado isogonal do segundo ponto de Fermat-Torricelli (problema $10$) com respeito a $\triangle ABC$. \dica %Use coordenadas baricêntricas %Determine as coordenadas baricêntricas do conjugado isogonal de um ponto de coordenadas baricêntricas $(u,v,w)$ %Faça uma razão entre as coordenadas baricêntricas dos pontos %Trigonometria
    
    \item \iniTri. Seja $L$ seu ponto de Lemoine (problema $6$). Os pontos $M,K\in AB$, $H,I\in BC$ e $J,G\in AC$ são escolhidos de tal forma que $MI\parallel AC$, $GH\parallel AB$, $KJ\parallel BC$ e $MI\cap KJ\cap GH=L$. Prove que os pontos $M,K,H,I,J$ e $G$ ficam sobre uma circunferência. \dica %Procure quadriláteros cíclicos %Analise $L$ em $\triangle AKJ$ %Aplique Lei dos Senos
    
    \item \iniTri. Seja $L$ seu ponto de Lemoine (problema $6$). Os pontos $M,K\in AB$, $H,I\in BC$ e $J,G\in AC$ são escolhidos de tal modo que os quadriláteros $MICA$, $GHBA$ e $KJCB$ são cíclicos e $MI\cap KJ\cap GH=L$. Prove que os pontos $M,K,H,I,J$ e $G$ ficam sobre uma circunferência de centro $L$. \dica %Marque ângulos
    
    \item Seja $ABCD$ um quadrilátero convexo tal que $AB\cap CD=E$ e $AD\cap BC=E$. Prove que os circuncírculos de $\triangle BFC$, $\triangle AFD$ e $\triangle ABE$ passam por um ponto em comum. \dica %Ponto fantasma %Marque ângulos
    
    \item A construção do problema $33$ é dada. Prove que o ponto $M$ e os respectivos centros $O_1$, $O_2$, $O_3$ e $O_4$ dos circuncírculos de $\triangle AFD$, $\triangle BFC$, $\triangle ABE$ e $\triangle DCE$ ficam sobre uma circunferência. \dica %Marque ângulos %Busque uma boa rotohomotetia
    
    \item As circunferências $k_1,k_2,k_3$ e $k_4$ são dadas de tal modo que elas passam por um ponto em comum $M$. Prove que as circunferências que passam pelos pontos de interseção de $(k_1,k_2,k_3)$, $(k_1,k_2,k_4)$, $(k_1,k_3,k_4)$ e $(k_2,k_3,k_4)$, diferentes de $M$, também passam por um ponto em comum. \dica %Faça uma inversão num bom centro %Teorema de Miquel
    
    \item Seja $ABCDE$ um pentágono convexo tal que $AC\cap BE=D_1$, $BD\cap AC=E_1$, $BD\cap EC=A_1$, $EC\cap AD=B_1$ e $AD\cap BE=C_1$. Digamos que $(XYZ)$ denote o circuncírculo de $\triangle XYZ$. Sejam $(AD_1C_1)\cap(B_1C_1E)=\{C_1,C_2\}$, $(B_1C_1E)\cap(A_1B_1D)=\{B_1,B_2\}$, $(A_1B_1D)\cap(A_1E_1C)=\{A_1,A_2\}$, $(A_1E_1C)\cap(E_1D_1B)=\{E_1,E_2\}$ e $(E_1D_1B)\cap(C_1D_1A)=\{D_1,D_2\}$. Prove que os pontos $A_2$, $B_2$, $C_2$, $D_2$ e $E_2$ ficam sobre uma circunferência. \dica %Marque ângulos
    
    \item Seja $ABCDE$ um pentágono convexo tal que $AC\cap BE=D^{\prime}$, $BD\cap AC=E^{\prime}$, $BD\cap EC=A^{\prime}$, $EC\cap AD=B^{\prime}$ e $AD\cap BE=C^{\prime}$. Digamos que $(XYZ)$ denote o circuncírculo de $\triangle XYZ$. Sejam $(AD^{\prime}B)\cap(BE^{\prime}C)=\{B,B^{\prime\prime}\}$, $(BE^{\prime}C)\cap(CA^{\prime}D)=\{C,C^{\prime\prime}\}$, $(CA^{\prime}D)\cap(DB^{\prime}E)=\{D,D^{\prime\prime}\}$, $(DB^{\prime}E)\cap(AC^{\prime}E)=\{E,E^{\prime\prime}\}$ e $(AC^{\prime}E)\cap(AD^{\prime}B)=\{A,A^{\prime\prime}\}$. Prove que as retas $AA^{\prime\prime},BB^{\prime\prime},CC^{\prime\prime},DD^{\prime\prime}$ e $EE^{\prime\prime}$ são concorrentes. \dica %Eixos radicais %Marque ângulos
    
    
    
    
    
    
    
    
    
    
    
    
    
    
    
    
    
    
    
\end{enumerate}
\end{multicols}


\end{document}























