\documentclass{article}
\usepackage[brazil]{babel}
\usepackage[utf8]{inputenc}
\usepackage[T1]{fontenc}
\usepackage{lmodern}
\usepackage{soulutf8} 
\setlength{\headheight}{14.5pt}
\usepackage[a4paper,left=2.0cm,right=1.0cm,top=\dimexpr15mm+1.5\baselineskip,bottom=2cm]{geometry}
\usepackage{amsmath}
\usepackage{tikz}
\usepackage{mathdots}
\usepackage{yhmath}
\usepackage{cancel}
\usepackage{color}
\usepackage{wasysym}
\usepackage{siunitx}
\usepackage{array}
\usepackage{xargs}
\usepackage{multirow}
\usepackage{amssymb}
\usepackage{tabularx}
\usepackage{extarrows}
\usepackage{xfrac}
\usepackage{booktabs}
\usetikzlibrary{fadings}
\usetikzlibrary{patterns}
\usetikzlibrary{shadows.blur}
\usetikzlibrary{shapes}
\usepackage[utf8]{inputenc}
\usepackage{amsfonts}
\usepackage{amsmath}
\usepackage{multicol}
\usepackage{fancyhdr}
\usepackage{tikz, tkz-euclide}
\usetikzlibrary{patterns}
\usepackage{enumitem}
\usepackage{arcs}
\usepackage{afterpage}
\usepackage{tcolorbox}
\usepackage{multirow}
\usepackage{numprint}
\usepackage{titlesec}
\usepackage{graphicx}
\usepackage{pgfplots}
\pgfplotsset{compat=1.15}
\usetikzlibrary{patterns}
\pgfplotsset{compat=1.15}
\usepackage{mathrsfs}
\usetikzlibrary{calc,arrows,matrix}
\usetikzlibrary{shapes.geometric}
\usetikzlibrary{positioning,quotes}
\definecolor{BLA}{rgb}{0.26666666666666666,0.26666666666666666,0.26666666666666666}
\definecolor{WHI}{rgb}{255,255,255}

\titleformat{\section}[block]{\Large\bfseries\filcenter}{\thesection}{1em}{\ul}
\titleformat{\subsection}{\large\bfseries\filcenter}{\thesubsection}{1em}{}

\newcommand{\limite}[3][x]{\lim\limits_{{#1}\to{#2}}{#3}}

\newcommand{\sgn}{\hspace{2pt}\mathrm{sgn}\hspace{1pt}}

\newcommand{\pint}[1]{\lfloor{#1}\rfloor}

\newcommand{\deriv}[1]{{#1}^{\prime}}
\newcommand{\segderiv}[1]{{#1}^{\prime\prime}}

\newcommand{\integ}[3]{\int\limits_{#1}^{#2}{#3}}

\newcommand{\fimd}[1]{\hspace{2pt}d{#1}}

\newcommand{\multi}[7]{\textbf{({#1})} {#2}
\begin{enumerate}
    \item {#3}
    
    \item {#4}
    
    \item {#5}
    
    \item {#6}
    
    \item {#7}
\end{enumerate}}

\newcommand{\parenth}[1]{\left({#1}\right)}

\newcommand{\opcaoIMG}[2][0.5]{}

\pgfdeclarepatternformonly{south west lines}{\pgfqpoint{-0pt}{-0pt}}{\pgfqpoint{6pt}{6pt}}{\pgfqpoint{6pt}{6pt}}{
        \pgfsetlinewidth{0.16pt}
        \pgfpathmoveto{\pgfqpoint{0pt}{0pt}}
        \pgfpathlineto{\pgfqpoint{6pt}{6pt}}
        \pgfpathmoveto{\pgfqpoint{5.6pt}{-0.4pt}}
        \pgfpathlineto{\pgfqpoint{6.4pt}{0.4pt}}
        \pgfpathmoveto{\pgfqpoint{-0.4pt}{5.6pt}}
        \pgfpathlineto{\pgfqpoint{0.4pt}{6.4pt}}
        \pgfusepath{stroke}}
        
\makeatletter
\DeclareFontFamily{U}{tipa}{}
\DeclareFontShape{U}{tipa}{m}{n}{<->tipa10}{}
\newcommand{\arc@char}{{\usefont{U}{tipa}{m}{n}\symbol{62}}}%

\newcommand{\arc}[1]{\mathpalette\arc@arc{#1}}

\newcommand{\arc@arc}[2]{
	\sbox0{$\m@th#1#2$}%
	\vbox{
		\hbox{\resizebox{\wd0}{\height}{\arc@char}}
		\nointerlineskip
		\box0
	}%
}
\makeatother

\DeclareFontFamily{U}{skulls}{}
\DeclareFontShape{U}{skulls}{m}{n}{ <-> skull }{}
\newcommand{\skull}{\text{\usefont{U}{skulls}{m}{n}\symbol{'101}}}

\pagestyle{fancy}
\fancyhf{}
\fancyhfoffset[L]{0.5cm} % left extra length
\fancyhfoffset[R]{0.5cm} % right extra length
\lhead{MATH CHRONIC}
\chead{\bfseries 555 problemas de Geometria}
\rhead{Jeferson Almir}
\rfoot{}
\cfoot{\thepage}

\providecommand{\sin}{}
\renewcommand{\sin}{\hspace{2pt}\mathrm{sen}\hspace{1pt}}
\providecommand{\tan}{}
\renewcommand{\tan}{\hspace{2pt}\mathrm{tg}\hspace{1pt}}
\providecommand{\cot}{}
\renewcommand{\cot}{\hspace{2pt}\mathrm{cotg}\hspace{1pt}}
\providecommand{\csc}{}
\renewcommand{\csc}{\hspace{2pt}\mathrm{cossec}\hspace{1pt}}
\providecommand{\arctan}{}
\renewcommand{\arctan}{\hspace{2pt}\mathrm{arctg}\hspace{1pt}}
\providecommand{\arcsin}{}
\renewcommand{\arcsin}{\hspace{2pt}\mathrm{arcsen}\hspace{1pt}}
\newcommand{\arccot}{\hspace{2pt}\mathrm{arccotg}\hspace{1pt}}

\newcommand{\espaco}{$ $}
\newcommand{\dica}{\textbf{Dicas:}}

\newcommand{\iniTri}{Seja $ABC$ um triângulo}

\definecolor{ffqqqq}{rgb}{1.,0.,0.}
\definecolor{qqqqff}{rgb}{0.,0.,1.}
  
  
  
\begin{document}
\begin{center}
Professor: Jeferson Almir
\end{center}

Aluno(a): \underline{\hspace{12cm}}
Nº: \underline{\hspace{2cm}}

Data: \underline{\hspace{2cm}}/\underline{\hspace{2cm}}/\underline{\hspace{2cm}}

\begin{multicols}{2}

\section{Problemas}

\renewcommand{\labelenumi}{\textbf{\arabic{enumi}.}}
\renewcommand\labelenumi{\textbf{\nplpadding{3}\numprint{\arabic{enumi}}.}}

\begin{enumerate}

    \item Seja $ABC$ um triângulo. Prove que suas medianas $CD$, $AE$ e $BF$ são concorrentes. \dica %Áreas %Proporção %Ponto fantasma
    
    \item Seja $ABC$ um triângulo. Prove que suas alturas $AE$, $CF$ e $BD$ são concorrentes. \dica %Quadriláteros cíclicos %Ponto fantasma
    
    \item Prove que as bissetrizes internas de um $\triangle ABC$ são concorrentes. \dica %Definição
    
    \item Seja $ABC$ um triângulo. Seu incírculo toca $AB$, $BC$ e $CA$ nos pontos $C_1$, $A_1$ e $B_1$ respectivamente. Prove que as retas $CC_1$, $BB_1$ e $AA_1$ são concorrentes. \dica %Teorema de Ceva
    
    \item Prove que as mediatrizes dos lados de um dado $\triangle ABC$ são concorrentes. \dica %Definição
    
    \item Seja $ABC$ um triângulo de circuncírculo $k$. Sejam $l_A$, $l_B$ e $l_C$ as retas tangentes a $k$ pelos pontos $A$, $B$ e $C$ respectivamente. Se $l_A\cap l_B=C_1$, $l_B\cap l_C=A_1$ e $l_C\cap l_A=B_1$, prove que as retas $AA_1$, $BB_1$ e $CC_1$ são concorrentes. \dica %Teorema de Ceva %Analise outro triângulo
    
    \item Seja $ABC$ um triângulo. Sejam $A_1$, $B_1$ e $C_1$ os pontos de tangência dos segmentos $BC$, $CA$ e $AB$ com os exincírculos de $\triangle ABC$. Prove que as retas $AA_1$, $BB_1$ e $CC_1$ são concorrentes. \dica %Teorema do Bico %Teorema de Ceva
    
    \item Seja $ABC$ um triângulo e seja $N$ seu ponto de Nagel (ponto de concorrência do exercício anterior). Digamos que $AN$, $BN$ e $CN$ intersectem o incírculo de $\triangle ABC$ nos pontos $A_1$, $B_1$ e $C_1$, e os lados $BC$, $CA$ e $AB$ nos pontos $A_2$, $B_2$ e $C_2$, respectivamente. Prove que $AA_1=NA_2$, $BB_1=NB_2$ e $CC_1=NC_2$. \dica %Ponto fantasma %Tente achar uma homotetia útil %Teorema de Menelaus
    
    \item Seja $ABC$ um triângulo. Os triângulos equiláteros $\triangle ABC_1$, $\triangle AB_1C$ e $\triangle A_1BC$ são construídos no exterior do triângulo $ABC$. Prove que as retas $AA_1$, $BB_1$ e $CC_1$ são concorrentes. \dica %Marque ângulos %Procure quadriláteros cíclicos %Prove colinearidade
    
    \item \iniTri. Os triângulos equiláteros $\triangle ABC_1$, $\triangle AB_1C$ e $\triangle A_1BC$ são construídos no interior do triângulo $ABC$. Prove que as retas $AA_1$, $BB_1$ e $CC_1$ são concorrentes. \dica %Marque ângulos %Procure quadriláteros cíclicos %Prove colinearidade
    
    
    
    
    
    
    
    
    
    
    
    
    
    
    
    
    
    
    
\end{enumerate}
\end{multicols}


\end{document}























